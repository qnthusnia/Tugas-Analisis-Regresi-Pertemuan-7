% Options for packages loaded elsewhere
\PassOptionsToPackage{unicode}{hyperref}
\PassOptionsToPackage{hyphens}{url}
%
\documentclass[
]{article}
\usepackage{amsmath,amssymb}
\usepackage{iftex}
\ifPDFTeX
  \usepackage[T1]{fontenc}
  \usepackage[utf8]{inputenc}
  \usepackage{textcomp} % provide euro and other symbols
\else % if luatex or xetex
  \usepackage{unicode-math} % this also loads fontspec
  \defaultfontfeatures{Scale=MatchLowercase}
  \defaultfontfeatures[\rmfamily]{Ligatures=TeX,Scale=1}
\fi
\usepackage{lmodern}
\ifPDFTeX\else
  % xetex/luatex font selection
\fi
% Use upquote if available, for straight quotes in verbatim environments
\IfFileExists{upquote.sty}{\usepackage{upquote}}{}
\IfFileExists{microtype.sty}{% use microtype if available
  \usepackage[]{microtype}
  \UseMicrotypeSet[protrusion]{basicmath} % disable protrusion for tt fonts
}{}
\makeatletter
\@ifundefined{KOMAClassName}{% if non-KOMA class
  \IfFileExists{parskip.sty}{%
    \usepackage{parskip}
  }{% else
    \setlength{\parindent}{0pt}
    \setlength{\parskip}{6pt plus 2pt minus 1pt}}
}{% if KOMA class
  \KOMAoptions{parskip=half}}
\makeatother
\usepackage{xcolor}
\usepackage[margin=1in]{geometry}
\usepackage{color}
\usepackage{fancyvrb}
\newcommand{\VerbBar}{|}
\newcommand{\VERB}{\Verb[commandchars=\\\{\}]}
\DefineVerbatimEnvironment{Highlighting}{Verbatim}{commandchars=\\\{\}}
% Add ',fontsize=\small' for more characters per line
\usepackage{framed}
\definecolor{shadecolor}{RGB}{248,248,248}
\newenvironment{Shaded}{\begin{snugshade}}{\end{snugshade}}
\newcommand{\AlertTok}[1]{\textcolor[rgb]{0.94,0.16,0.16}{#1}}
\newcommand{\AnnotationTok}[1]{\textcolor[rgb]{0.56,0.35,0.01}{\textbf{\textit{#1}}}}
\newcommand{\AttributeTok}[1]{\textcolor[rgb]{0.13,0.29,0.53}{#1}}
\newcommand{\BaseNTok}[1]{\textcolor[rgb]{0.00,0.00,0.81}{#1}}
\newcommand{\BuiltInTok}[1]{#1}
\newcommand{\CharTok}[1]{\textcolor[rgb]{0.31,0.60,0.02}{#1}}
\newcommand{\CommentTok}[1]{\textcolor[rgb]{0.56,0.35,0.01}{\textit{#1}}}
\newcommand{\CommentVarTok}[1]{\textcolor[rgb]{0.56,0.35,0.01}{\textbf{\textit{#1}}}}
\newcommand{\ConstantTok}[1]{\textcolor[rgb]{0.56,0.35,0.01}{#1}}
\newcommand{\ControlFlowTok}[1]{\textcolor[rgb]{0.13,0.29,0.53}{\textbf{#1}}}
\newcommand{\DataTypeTok}[1]{\textcolor[rgb]{0.13,0.29,0.53}{#1}}
\newcommand{\DecValTok}[1]{\textcolor[rgb]{0.00,0.00,0.81}{#1}}
\newcommand{\DocumentationTok}[1]{\textcolor[rgb]{0.56,0.35,0.01}{\textbf{\textit{#1}}}}
\newcommand{\ErrorTok}[1]{\textcolor[rgb]{0.64,0.00,0.00}{\textbf{#1}}}
\newcommand{\ExtensionTok}[1]{#1}
\newcommand{\FloatTok}[1]{\textcolor[rgb]{0.00,0.00,0.81}{#1}}
\newcommand{\FunctionTok}[1]{\textcolor[rgb]{0.13,0.29,0.53}{\textbf{#1}}}
\newcommand{\ImportTok}[1]{#1}
\newcommand{\InformationTok}[1]{\textcolor[rgb]{0.56,0.35,0.01}{\textbf{\textit{#1}}}}
\newcommand{\KeywordTok}[1]{\textcolor[rgb]{0.13,0.29,0.53}{\textbf{#1}}}
\newcommand{\NormalTok}[1]{#1}
\newcommand{\OperatorTok}[1]{\textcolor[rgb]{0.81,0.36,0.00}{\textbf{#1}}}
\newcommand{\OtherTok}[1]{\textcolor[rgb]{0.56,0.35,0.01}{#1}}
\newcommand{\PreprocessorTok}[1]{\textcolor[rgb]{0.56,0.35,0.01}{\textit{#1}}}
\newcommand{\RegionMarkerTok}[1]{#1}
\newcommand{\SpecialCharTok}[1]{\textcolor[rgb]{0.81,0.36,0.00}{\textbf{#1}}}
\newcommand{\SpecialStringTok}[1]{\textcolor[rgb]{0.31,0.60,0.02}{#1}}
\newcommand{\StringTok}[1]{\textcolor[rgb]{0.31,0.60,0.02}{#1}}
\newcommand{\VariableTok}[1]{\textcolor[rgb]{0.00,0.00,0.00}{#1}}
\newcommand{\VerbatimStringTok}[1]{\textcolor[rgb]{0.31,0.60,0.02}{#1}}
\newcommand{\WarningTok}[1]{\textcolor[rgb]{0.56,0.35,0.01}{\textbf{\textit{#1}}}}
\usepackage{graphicx}
\makeatletter
\def\maxwidth{\ifdim\Gin@nat@width>\linewidth\linewidth\else\Gin@nat@width\fi}
\def\maxheight{\ifdim\Gin@nat@height>\textheight\textheight\else\Gin@nat@height\fi}
\makeatother
% Scale images if necessary, so that they will not overflow the page
% margins by default, and it is still possible to overwrite the defaults
% using explicit options in \includegraphics[width, height, ...]{}
\setkeys{Gin}{width=\maxwidth,height=\maxheight,keepaspectratio}
% Set default figure placement to htbp
\makeatletter
\def\fps@figure{htbp}
\makeatother
\setlength{\emergencystretch}{3em} % prevent overfull lines
\providecommand{\tightlist}{%
  \setlength{\itemsep}{0pt}\setlength{\parskip}{0pt}}
\setcounter{secnumdepth}{-\maxdimen} % remove section numbering
\ifLuaTeX
  \usepackage{selnolig}  % disable illegal ligatures
\fi
\IfFileExists{bookmark.sty}{\usepackage{bookmark}}{\usepackage{hyperref}}
\IfFileExists{xurl.sty}{\usepackage{xurl}}{} % add URL line breaks if available
\urlstyle{same}
\hypersetup{
  pdftitle={Tugas Analisis Regresi Pertemuan 7},
  pdfauthor={Qonita Husnia Rahmah},
  hidelinks,
  pdfcreator={LaTeX via pandoc}}

\title{Tugas Analisis Regresi Pertemuan 7}
\author{Qonita Husnia Rahmah}
\date{2024-03-05}

\begin{document}
\maketitle

\hypertarget{library}{%
\subsection{Library}\label{library}}

\begin{Shaded}
\begin{Highlighting}[]
\FunctionTok{library}\NormalTok{(tidyverse)}
\end{Highlighting}
\end{Shaded}

\begin{verbatim}
## Warning: package 'tidyverse' was built under R version 4.3.2
\end{verbatim}

\begin{verbatim}
## Warning: package 'ggplot2' was built under R version 4.3.2
\end{verbatim}

\begin{verbatim}
## Warning: package 'dplyr' was built under R version 4.3.2
\end{verbatim}

\begin{verbatim}
## Warning: package 'lubridate' was built under R version 4.3.2
\end{verbatim}

\begin{verbatim}
## -- Attaching core tidyverse packages ------------------------ tidyverse 2.0.0 --
## v dplyr     1.1.4     v readr     2.1.4
## v forcats   1.0.0     v stringr   1.5.0
## v ggplot2   3.4.4     v tibble    3.2.1
## v lubridate 1.9.3     v tidyr     1.3.0
## v purrr     1.0.2     
## -- Conflicts ------------------------------------------ tidyverse_conflicts() --
## x dplyr::filter() masks stats::filter()
## x dplyr::lag()    masks stats::lag()
## i Use the conflicted package (<http://conflicted.r-lib.org/>) to force all conflicts to become errors
\end{verbatim}

\begin{Shaded}
\begin{Highlighting}[]
\FunctionTok{library}\NormalTok{(ggridges)}
\FunctionTok{library}\NormalTok{(GGally)}
\end{Highlighting}
\end{Shaded}

\begin{verbatim}
## Warning: package 'GGally' was built under R version 4.3.2
\end{verbatim}

\begin{verbatim}
## Registered S3 method overwritten by 'GGally':
##   method from   
##   +.gg   ggplot2
\end{verbatim}

\begin{Shaded}
\begin{Highlighting}[]
\FunctionTok{library}\NormalTok{(dplyr)}
\FunctionTok{library}\NormalTok{(lmtest)}
\end{Highlighting}
\end{Shaded}

\begin{verbatim}
## Warning: package 'lmtest' was built under R version 4.3.3
\end{verbatim}

\begin{verbatim}
## Loading required package: zoo
\end{verbatim}

\begin{verbatim}
## Warning: package 'zoo' was built under R version 4.3.3
\end{verbatim}

\begin{verbatim}
## 
## Attaching package: 'zoo'
## 
## The following objects are masked from 'package:base':
## 
##     as.Date, as.Date.numeric
\end{verbatim}

\begin{Shaded}
\begin{Highlighting}[]
\FunctionTok{library}\NormalTok{(stats)}
\end{Highlighting}
\end{Shaded}

\hypertarget{data}{%
\subsection{Data}\label{data}}

\begin{Shaded}
\begin{Highlighting}[]
\FunctionTok{library}\NormalTok{(readxl)}
\NormalTok{DataRegresi }\OtherTok{\textless{}{-}} \FunctionTok{read\_xlsx}\NormalTok{(}\StringTok{"C:}\SpecialCharTok{\textbackslash{}\textbackslash{}}\StringTok{Users}\SpecialCharTok{\textbackslash{}\textbackslash{}}\StringTok{ASUS}\SpecialCharTok{\textbackslash{}\textbackslash{}}\StringTok{Documents}\SpecialCharTok{\textbackslash{}\textbackslash{}}\StringTok{Nita}\SpecialCharTok{\textbackslash{}\textbackslash{}}\StringTok{SEMESTER 4}\SpecialCharTok{\textbackslash{}\textbackslash{}}\StringTok{Analisis Regresi}\SpecialCharTok{\textbackslash{}\textbackslash{}}\StringTok{Kuliah}\SpecialCharTok{\textbackslash{}\textbackslash{}}\StringTok{Data Tugas Pertemuan 7.xlsx"}\NormalTok{)}
\NormalTok{DataRegresi}
\end{Highlighting}
\end{Shaded}

\begin{verbatim}
## # A tibble: 15 x 3
##       No     X     Y
##    <dbl> <dbl> <dbl>
##  1     1     2    54
##  2     2     5    50
##  3     3     7    45
##  4     4    10    37
##  5     5    14    35
##  6     6    19    25
##  7     7    26    20
##  8     8    31    16
##  9     9    34    18
## 10    10    38    13
## 11    11    45     8
## 12    12    52    11
## 13    13    53     8
## 14    14    60     4
## 15    15    65     6
\end{verbatim}

\begin{Shaded}
\begin{Highlighting}[]
\NormalTok{Y}\OtherTok{\textless{}{-}}\NormalTok{DataRegresi}\SpecialCharTok{$}\NormalTok{Y}
\NormalTok{X}\OtherTok{\textless{}{-}}\NormalTok{DataRegresi}\SpecialCharTok{$}\NormalTok{X}

\NormalTok{data}\OtherTok{\textless{}{-}}\FunctionTok{data.frame}\NormalTok{(}\FunctionTok{cbind}\NormalTok{(X, Y))}
\FunctionTok{head}\NormalTok{(data)}
\end{Highlighting}
\end{Shaded}

\begin{verbatim}
##    X  Y
## 1  2 54
## 2  5 50
## 3  7 45
## 4 10 37
## 5 14 35
## 6 19 25
\end{verbatim}

\hypertarget{scatter-plot-antara-x-dan-y}{%
\subsubsection{Scatter Plot Antara X dan
Y}\label{scatter-plot-antara-x-dan-y}}

\begin{Shaded}
\begin{Highlighting}[]
\FunctionTok{plot}\NormalTok{(X,Y)}
\end{Highlighting}
\end{Shaded}

\includegraphics{Tugas-Analisis-Regresi-Pertemuan-7_files/figure-latex/unnamed-chunk-4-1.pdf}

Dari scatter plot antara X dan Y, dapat dilihat bahwa X dan Y memiliki
hubungan yang tidak linear cenderung membentuk parabola.

\hypertarget{pemodel-regresi-linear}{%
\subsubsection{Pemodel Regresi Linear}\label{pemodel-regresi-linear}}

\begin{Shaded}
\begin{Highlighting}[]
\NormalTok{model }\OtherTok{\textless{}{-}} \FunctionTok{lm}\NormalTok{(}\AttributeTok{formula =}\NormalTok{ Y }\SpecialCharTok{\textasciitilde{}}\NormalTok{ X, }\AttributeTok{data =}\NormalTok{ data)}
\FunctionTok{summary}\NormalTok{(model)}
\end{Highlighting}
\end{Shaded}

\begin{verbatim}
## 
## Call:
## lm(formula = Y ~ X, data = data)
## 
## Residuals:
##     Min      1Q  Median      3Q     Max 
## -7.1628 -4.7313 -0.9253  3.7386  9.0446 
## 
## Coefficients:
##             Estimate Std. Error t value Pr(>|t|)    
## (Intercept) 46.46041    2.76218   16.82 3.33e-10 ***
## X           -0.75251    0.07502  -10.03 1.74e-07 ***
## ---
## Signif. codes:  0 '***' 0.001 '**' 0.01 '*' 0.05 '.' 0.1 ' ' 1
## 
## Residual standard error: 5.891 on 13 degrees of freedom
## Multiple R-squared:  0.8856, Adjusted R-squared:  0.8768 
## F-statistic: 100.6 on 1 and 13 DF,  p-value: 1.736e-07
\end{verbatim}

\begin{Shaded}
\begin{Highlighting}[]
\NormalTok{model}
\end{Highlighting}
\end{Shaded}

\begin{verbatim}
## 
## Call:
## lm(formula = Y ~ X, data = data)
## 
## Coefficients:
## (Intercept)            X  
##     46.4604      -0.7525
\end{verbatim}

\hypertarget{eksplorasi-kondisi-gauss-markov-peneriksaan-dengan-grafik}{%
\section{Eksplorasi Kondisi Gauss-Markov, Peneriksaan dengan
Grafik}\label{eksplorasi-kondisi-gauss-markov-peneriksaan-dengan-grafik}}

\hypertarget{plot-sisaan-vs-y-duga}{%
\subsection{Plot Sisaan vs Y duga}\label{plot-sisaan-vs-y-duga}}

\begin{Shaded}
\begin{Highlighting}[]
\FunctionTok{plot}\NormalTok{(model,}\DecValTok{1}\NormalTok{)}
\end{Highlighting}
\end{Shaded}

\includegraphics{Tugas-Analisis-Regresi-Pertemuan-7_files/figure-latex/unnamed-chunk-7-1.pdf}

\hypertarget{plot-sisaan-vs-urutan}{%
\subsection{Plot Sisaan vs Urutan}\label{plot-sisaan-vs-urutan}}

\begin{Shaded}
\begin{Highlighting}[]
 \FunctionTok{plot}\NormalTok{(}\AttributeTok{x =} \DecValTok{1}\SpecialCharTok{:}\FunctionTok{dim}\NormalTok{(data)[}\DecValTok{1}\NormalTok{],}
 \AttributeTok{y =}\NormalTok{ model}\SpecialCharTok{$}\NormalTok{residuals,}
 \AttributeTok{type =} \StringTok{\textquotesingle{}b\textquotesingle{}}\NormalTok{,}
 \AttributeTok{ylab =} \StringTok{"Residuals"}\NormalTok{,}
 \AttributeTok{xlab =} \StringTok{"Observation"}\NormalTok{)}
\end{Highlighting}
\end{Shaded}

\includegraphics{Tugas-Analisis-Regresi-Pertemuan-7_files/figure-latex/unnamed-chunk-8-1.pdf}

Sebaran membentuk pola kurva, artinya Sisaan tidak saling bebas

\hypertarget{normalitas-sisaan-dengan-qq-plot}{%
\subsection{Normalitas Sisaan dengan
QQ-Plot}\label{normalitas-sisaan-dengan-qq-plot}}

\begin{Shaded}
\begin{Highlighting}[]
\FunctionTok{plot}\NormalTok{(model,}\DecValTok{2}\NormalTok{)}
\end{Highlighting}
\end{Shaded}

\includegraphics{Tugas-Analisis-Regresi-Pertemuan-7_files/figure-latex/unnamed-chunk-9-1.pdf}

\hypertarget{pengujian-asumsi}{%
\section{Pengujian Asumsi}\label{pengujian-asumsi}}

\hypertarget{uji-asumsi-gauss-markov}{%
\subsection{Uji Asumsi Gauss-Markov}\label{uji-asumsi-gauss-markov}}

\hypertarget{nilai-harapan-sisaan-sama-dengan-nol}{%
\subsubsection{1. Nilai Harapan Sisaan sama dengan
nol}\label{nilai-harapan-sisaan-sama-dengan-nol}}

\[
H_0 : \text{Nilai harapan sama dengan 0}\\H_1 : \text{Nilai harapan tidak sama dengan 0}
\]

\begin{Shaded}
\begin{Highlighting}[]
\FunctionTok{t.test}\NormalTok{(model}\SpecialCharTok{$}\NormalTok{residuals,}\AttributeTok{mu =} \DecValTok{0}\NormalTok{,}\AttributeTok{conf.level =} \FloatTok{0.95}\NormalTok{)}
\end{Highlighting}
\end{Shaded}

\begin{verbatim}
## 
##  One Sample t-test
## 
## data:  model$residuals
## t = -4.9493e-16, df = 14, p-value = 1
## alternative hypothesis: true mean is not equal to 0
## 95 percent confidence interval:
##  -3.143811  3.143811
## sample estimates:
##     mean of x 
## -7.254614e-16
\end{verbatim}

Dalam pengujian asumsi Gauss-Markov yang pertama, didapat bahwa p-value
\textgreater{} 0.05, maka tak tolak \(H_0\), ataun nilai harapan sisaan
sama dengan nol

\hypertarget{ragam-sisaan-homogen-homogenitas}{%
\subsubsection{2. Ragam Sisaan Homogen
(Homogenitas)}\label{ragam-sisaan-homogen-homogenitas}}

\[
H_0 : \text{Ragam sisaan homogen}\\H_1 : \text{Ragam sisaan tidak homogen}
\]

\begin{Shaded}
\begin{Highlighting}[]
\NormalTok{kehomogenan }\OtherTok{=} \FunctionTok{lm}\NormalTok{(}\AttributeTok{formula =} \FunctionTok{abs}\NormalTok{(model}\SpecialCharTok{$}\NormalTok{residuals) }\SpecialCharTok{\textasciitilde{}}\NormalTok{ X, }
\AttributeTok{data =}\NormalTok{ data)}
\FunctionTok{summary}\NormalTok{(kehomogenan)}
\end{Highlighting}
\end{Shaded}

\begin{verbatim}
## 
## Call:
## lm(formula = abs(model$residuals) ~ X, data = data)
## 
## Residuals:
##     Min      1Q  Median      3Q     Max 
## -4.2525 -1.7525  0.0235  2.0168  4.2681 
## 
## Coefficients:
##             Estimate Std. Error t value Pr(>|t|)    
## (Intercept)  5.45041    1.27241   4.284  0.00089 ***
## X           -0.01948    0.03456  -0.564  0.58266    
## ---
## Signif. codes:  0 '***' 0.001 '**' 0.01 '*' 0.05 '.' 0.1 ' ' 1
## 
## Residual standard error: 2.714 on 13 degrees of freedom
## Multiple R-squared:  0.02385,    Adjusted R-squared:  -0.05124 
## F-statistic: 0.3176 on 1 and 13 DF,  p-value: 0.5827
\end{verbatim}

\begin{Shaded}
\begin{Highlighting}[]
\FunctionTok{bptest}\NormalTok{(model)}
\end{Highlighting}
\end{Shaded}

\begin{verbatim}
## 
##  studentized Breusch-Pagan test
## 
## data:  model
## BP = 0.52819, df = 1, p-value = 0.4674
\end{verbatim}

Didapat bahwa p-value \textgreater{} 0.05, maka taktolak \(H_0\), atau
ragam sisaan homogen untk setiap nilai x.

\hypertarget{sisaan-saling-bebastidak-ada-autokorelasi}{%
\subsubsection{3. Sisaan Saling Bebas/Tidak Ada
Autokorelasi}\label{sisaan-saling-bebastidak-ada-autokorelasi}}

\[
H_0 : \text{Sisaan saling bebas}\\H_1 : \text{Sisaan tidak saling bebas}
\]

\begin{Shaded}
\begin{Highlighting}[]
\FunctionTok{library}\NormalTok{ (randtests)}
\FunctionTok{runs.test}\NormalTok{(model}\SpecialCharTok{$}\NormalTok{residuals)}
\end{Highlighting}
\end{Shaded}

\begin{verbatim}
## 
##  Runs Test
## 
## data:  model$residuals
## statistic = -2.7817, runs = 3, n1 = 7, n2 = 7, n = 14, p-value =
## 0.005407
## alternative hypothesis: nonrandomness
\end{verbatim}

\begin{Shaded}
\begin{Highlighting}[]
\FunctionTok{dwtest}\NormalTok{(model)}
\end{Highlighting}
\end{Shaded}

\begin{verbatim}
## 
##  Durbin-Watson test
## 
## data:  model
## DW = 0.48462, p-value = 1.333e-05
## alternative hypothesis: true autocorrelation is greater than 0
\end{verbatim}

Didapat bahwa p-value \textless{} 0.05, maka tolak \(H_0\), atau sisaan
tidak saling bebas(terdapat autokorelasi)

\hypertarget{uji-normalitas-sisaan}{%
\subsection{Uji Normalitas Sisaan}\label{uji-normalitas-sisaan}}

\[
H_0 : \text{Sisaan menyebar normal}\\H_1 : \text{Sisaan tidak menyebar normal}
\]

\begin{Shaded}
\begin{Highlighting}[]
\FunctionTok{shapiro.test}\NormalTok{(model}\SpecialCharTok{$}\NormalTok{residuals)}
\end{Highlighting}
\end{Shaded}

\begin{verbatim}
## 
##  Shapiro-Wilk normality test
## 
## data:  model$residuals
## W = 0.92457, p-value = 0.226
\end{verbatim}

Dari Uji Shapiro-Wilk, didapat hasil bahwa p-value \textgreater{} alpha,
maka tak tolak \(H_0\), atau sisaan menyebar normal dengan alpha 0.05.

\hypertarget{kesimpulan}{%
\subsection{Kesimpulan}\label{kesimpulan}}

Dari ketiga asumsi Gauss-Markov yang ada, didapat hasil bahwa terdapat
asumsi yang tidak terpenuhi yaitu pelanggaran asumsi tidak ada
autokorelasi dengan melihat Durbin Watson Test yang telah dilakukan.

\hypertarget{penanganan-kondisi-tak-standar}{%
\section{Penanganan Kondisi Tak
Standar}\label{penanganan-kondisi-tak-standar}}

\hypertarget{transformasi-data}{%
\subsection{Transformasi Data}\label{transformasi-data}}

\begin{Shaded}
\begin{Highlighting}[]
\NormalTok{new\_x }\OtherTok{\textless{}{-}} \FunctionTok{sqrt}\NormalTok{(data}\SpecialCharTok{$}\NormalTok{X)}
\NormalTok{new\_y }\OtherTok{\textless{}{-}} \FunctionTok{sqrt}\NormalTok{(data}\SpecialCharTok{$}\NormalTok{Y)}
\NormalTok{transformasi }\OtherTok{\textless{}{-}} \FunctionTok{data.frame}\NormalTok{(new\_x, new\_y)}
\NormalTok{transformasi}
\end{Highlighting}
\end{Shaded}

\begin{verbatim}
##       new_x    new_y
## 1  1.414214 7.348469
## 2  2.236068 7.071068
## 3  2.645751 6.708204
## 4  3.162278 6.082763
## 5  3.741657 5.916080
## 6  4.358899 5.000000
## 7  5.099020 4.472136
## 8  5.567764 4.000000
## 9  5.830952 4.242641
## 10 6.164414 3.605551
## 11 6.708204 2.828427
## 12 7.211103 3.316625
## 13 7.280110 2.828427
## 14 7.745967 2.000000
## 15 8.062258 2.449490
\end{verbatim}

\hypertarget{plot-baru}{%
\subsubsection{Plot Baru}\label{plot-baru}}

\begin{Shaded}
\begin{Highlighting}[]
\FunctionTok{plot}\NormalTok{(transformasi}\SpecialCharTok{$}\NormalTok{new\_x, transformasi}\SpecialCharTok{$}\NormalTok{new\_y)}
\end{Highlighting}
\end{Shaded}

\includegraphics{Tugas-Analisis-Regresi-Pertemuan-7_files/figure-latex/unnamed-chunk-16-1.pdf}

\hypertarget{model-linear-baru}{%
\subsubsection{Model Linear Baru}\label{model-linear-baru}}

\begin{Shaded}
\begin{Highlighting}[]
\NormalTok{ModelBaru }\OtherTok{\textless{}{-}} \FunctionTok{lm}\NormalTok{(transformasi}\SpecialCharTok{$}\NormalTok{new\_y }\SpecialCharTok{\textasciitilde{}}\NormalTok{ transformasi}\SpecialCharTok{$}\NormalTok{new\_x, }\AttributeTok{data=}\NormalTok{transformasi)}
\FunctionTok{summary}\NormalTok{ (ModelBaru)}
\end{Highlighting}
\end{Shaded}

\begin{verbatim}
## 
## Call:
## lm(formula = transformasi$new_y ~ transformasi$new_x, data = transformasi)
## 
## Residuals:
##      Min       1Q   Median       3Q      Max 
## -0.42765 -0.17534 -0.05753  0.21223  0.46960 
## 
## Coefficients:
##                    Estimate Std. Error t value Pr(>|t|)    
## (Intercept)         8.71245    0.19101   45.61 9.83e-16 ***
## transformasi$new_x -0.81339    0.03445  -23.61 4.64e-12 ***
## ---
## Signif. codes:  0 '***' 0.001 '**' 0.01 '*' 0.05 '.' 0.1 ' ' 1
## 
## Residual standard error: 0.2743 on 13 degrees of freedom
## Multiple R-squared:  0.9772, Adjusted R-squared:  0.9755 
## F-statistic: 557.3 on 1 and 13 DF,  p-value: 4.643e-12
\end{verbatim}

\begin{Shaded}
\begin{Highlighting}[]
\NormalTok{ModelBaru}
\end{Highlighting}
\end{Shaded}

\begin{verbatim}
## 
## Call:
## lm(formula = transformasi$new_y ~ transformasi$new_x, data = transformasi)
## 
## Coefficients:
##        (Intercept)  transformasi$new_x  
##             8.7125             -0.8134
\end{verbatim}

\hypertarget{eksplorasi-kondisi-gauss-markov-peneriksaan-dengan-grafik-1}{%
\section{Eksplorasi Kondisi Gauss-Markov, Peneriksaan dengan
Grafik}\label{eksplorasi-kondisi-gauss-markov-peneriksaan-dengan-grafik-1}}

\hypertarget{plot-sisaan-vs-y-duga-1}{%
\subsection{Plot Sisaan vs Y duga}\label{plot-sisaan-vs-y-duga-1}}

\begin{Shaded}
\begin{Highlighting}[]
\FunctionTok{plot}\NormalTok{(ModelBaru,}\DecValTok{1}\NormalTok{)}
\end{Highlighting}
\end{Shaded}

\includegraphics{Tugas-Analisis-Regresi-Pertemuan-7_files/figure-latex/unnamed-chunk-19-1.pdf}

\hypertarget{plot-sisaan-vs-urutan-1}{%
\subsection{Plot Sisaan vs Urutan}\label{plot-sisaan-vs-urutan-1}}

\begin{Shaded}
\begin{Highlighting}[]
 \FunctionTok{plot}\NormalTok{(}\AttributeTok{x =} \DecValTok{1}\SpecialCharTok{:}\FunctionTok{dim}\NormalTok{(transformasi)[}\DecValTok{1}\NormalTok{],}
 \AttributeTok{y =}\NormalTok{ ModelBaru}\SpecialCharTok{$}\NormalTok{residuals,}
 \AttributeTok{type =} \StringTok{\textquotesingle{}b\textquotesingle{}}\NormalTok{,}
 \AttributeTok{ylab =} \StringTok{"Residuals"}\NormalTok{,}
 \AttributeTok{xlab =} \StringTok{"Observation"}\NormalTok{)}
\end{Highlighting}
\end{Shaded}

\includegraphics{Tugas-Analisis-Regresi-Pertemuan-7_files/figure-latex/unnamed-chunk-20-1.pdf}

\hypertarget{normalitas-sisaan-dengan-qq-plot-1}{%
\subsection{Normalitas Sisaan dengan
QQ-Plot}\label{normalitas-sisaan-dengan-qq-plot-1}}

\begin{Shaded}
\begin{Highlighting}[]
\FunctionTok{plot}\NormalTok{(ModelBaru,}\DecValTok{2}\NormalTok{)}
\end{Highlighting}
\end{Shaded}

\includegraphics{Tugas-Analisis-Regresi-Pertemuan-7_files/figure-latex/unnamed-chunk-21-1.pdf}

\hypertarget{pengujian-asumsi-1}{%
\section{Pengujian Asumsi}\label{pengujian-asumsi-1}}

\hypertarget{uji-asumsi-gauss-markov-1}{%
\subsection{Uji Asumsi Gauss-Markov}\label{uji-asumsi-gauss-markov-1}}

\hypertarget{nilai-harapan-sisaan-sama-dengan-nol-1}{%
\subsubsection{1. Nilai Harapan Sisaan sama dengan
nol}\label{nilai-harapan-sisaan-sama-dengan-nol-1}}

\[
H_0 : \text{Nilai harapan sama dengan 0}\\H_1 : \text{Nilai harapan tidak sama dengan 0}
\]

\begin{Shaded}
\begin{Highlighting}[]
\FunctionTok{t.test}\NormalTok{(ModelBaru}\SpecialCharTok{$}\NormalTok{residuals,}\AttributeTok{mu =} \DecValTok{0}\NormalTok{,}\AttributeTok{conf.level =} \FloatTok{0.95}\NormalTok{)}
\end{Highlighting}
\end{Shaded}

\begin{verbatim}
## 
##  One Sample t-test
## 
## data:  ModelBaru$residuals
## t = 2.0334e-16, df = 14, p-value = 1
## alternative hypothesis: true mean is not equal to 0
## 95 percent confidence interval:
##  -0.1463783  0.1463783
## sample estimates:
##    mean of x 
## 1.387779e-17
\end{verbatim}

Dalam pengujian asumsi Gauss-Markov yang pertama, didapat bahwa p-value
\textgreater{} 0.05, maka tak tolak \(H_0\), ataun nilai harapan sisaan
sama dengan nol

\hypertarget{ragam-sisaan-homogen-homogenitas-1}{%
\subsubsection{2. Ragam Sisaan Homogen
(Homogenitas)}\label{ragam-sisaan-homogen-homogenitas-1}}

\[
H_0 : \text{Ragam sisaan homogen}\\H_1 : \text{Ragam sisaan tidak homogen}
\]

\begin{Shaded}
\begin{Highlighting}[]
\FunctionTok{library}\NormalTok{(car)}
\end{Highlighting}
\end{Shaded}

\begin{verbatim}
## Warning: package 'car' was built under R version 4.3.3
\end{verbatim}

\begin{verbatim}
## Loading required package: carData
\end{verbatim}

\begin{verbatim}
## 
## Attaching package: 'car'
\end{verbatim}

\begin{verbatim}
## The following object is masked from 'package:dplyr':
## 
##     recode
\end{verbatim}

\begin{verbatim}
## The following object is masked from 'package:purrr':
## 
##     some
\end{verbatim}

\begin{Shaded}
\begin{Highlighting}[]
\FunctionTok{ncvTest}\NormalTok{(ModelBaru)}
\end{Highlighting}
\end{Shaded}

\begin{verbatim}
## Non-constant Variance Score Test 
## Variance formula: ~ fitted.values 
## Chisquare = 2.160411, Df = 1, p = 0.14161
\end{verbatim}

Didapat bahwa p-value \textgreater{} 0.05, maka taktolak \(H_0\), atau
ragam sisaan homogen untk setiap nilai x.

\hypertarget{sisaan-saling-bebastidak-ada-autokorelasi-1}{%
\subsubsection{3. Sisaan Saling Bebas/Tidak Ada
Autokorelasi}\label{sisaan-saling-bebastidak-ada-autokorelasi-1}}

\[
H_0 : \text{Sisaan saling bebas}\\H_1 : \text{Sisaan tidak saling bebas}
\]

\begin{Shaded}
\begin{Highlighting}[]
\FunctionTok{library}\NormalTok{ (randtests)}
\FunctionTok{runs.test}\NormalTok{(ModelBaru}\SpecialCharTok{$}\NormalTok{residuals)}
\end{Highlighting}
\end{Shaded}

\begin{verbatim}
## 
##  Runs Test
## 
## data:  ModelBaru$residuals
## statistic = 0, runs = 8, n1 = 7, n2 = 7, n = 14, p-value = 1
## alternative hypothesis: nonrandomness
\end{verbatim}

\begin{Shaded}
\begin{Highlighting}[]
\FunctionTok{dwtest}\NormalTok{(model)}
\end{Highlighting}
\end{Shaded}

\begin{verbatim}
## 
##  Durbin-Watson test
## 
## data:  model
## DW = 0.48462, p-value = 1.333e-05
## alternative hypothesis: true autocorrelation is greater than 0
\end{verbatim}

Didapat bahwa p-value \textgreater{} 0.05, maka tolak \(H_0\), atau
sisaan saling bebas(tidak terdapat autokorelasi)

\hypertarget{uji-normalitas-sisaan-1}{%
\subsection{Uji Normalitas Sisaan}\label{uji-normalitas-sisaan-1}}

\[
H_0 : \text{Sisaan menyebar normal}\\H_1 : \text{Sisaan tidak menyebar normal}
\]

\begin{Shaded}
\begin{Highlighting}[]
\FunctionTok{shapiro.test}\NormalTok{(model}\SpecialCharTok{$}\NormalTok{residuals)}
\end{Highlighting}
\end{Shaded}

\begin{verbatim}
## 
##  Shapiro-Wilk normality test
## 
## data:  model$residuals
## W = 0.92457, p-value = 0.226
\end{verbatim}

Dari Uji Shapiro-Wilk, didapat hasil bahwa p-value \textgreater{} alpha,
maka tak tolak \(H_0\), atau sisaan menyebar normal dengan alpha 0.05.

\hypertarget{kesimpulan-1}{%
\subsection{Kesimpulan}\label{kesimpulan-1}}

Pada model baru, diperoleh hasil bahwa memenuhi ketiga asumsi
Gauss-markov serta sisaan menyebar normal. Berdasarkan transformasi yang
dilakukan, maka akan diproleh model regresi yang lebih efektif dengan
semua asumsi telah terpenuhi dalam analisis regresi linear sederhana.

Model regresi setelah di transformasi adalah sebagai berikut :
\[Y^*=8.71245-0.814X^* + e\] \[Y^* = \sqrt Y\] \[X^* = \sqrt X \]
Sehingga model terbaik untuk data ini adalah:
\[\hat Y=(8.71245-0.814X^\frac12)^2 + e\] Model regresi yang di peroleh
menunjukkan hubungan kuadrat negatif antara Y dan X. Ketika X meningkat,
Y cenderung akan menurun dengan kecepatan yang semakin cepat. Konstanta
sebesar 8.71245 mewakili nilai Y ketika X sama dengan 0. Koefisien
-0.0814 menunjukkan pengaruh perubahan X terhadap Y. Semakin besar nilai
absolut koefisien, semakin besar pengaruh X terhadap Y.

\end{document}
